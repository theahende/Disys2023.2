%%
% Copyright (c) 2017 - 2023, Pascal Wagler;
% Copyright (c) 2014 - 2023, John MacFarlane
%
% All rights reserved.
%
% Redistribution and use in source and binary forms, with or without
% modification, are permitted provided that the following conditions
% are met:
%
% - Redistributions of source code must retain the above copyright
% notice, this list of conditions and the following disclaimer.
%
% - Redistributions in binary form must reproduce the above copyright
% notice, this list of conditions and the following disclaimer in the
% documentation and/or other materials provided with the distribution.
%
% - Neither the name of John MacFarlane nor the names of other
% contributors may be used to endorse or promote products derived
% from this software without specific prior written permission.
%
% THIS SOFTWARE IS PROVIDED BY THE COPYRIGHT HOLDERS AND CONTRIBUTORS
% "AS IS" AND ANY EXPRESS OR IMPLIED WARRANTIES, INCLUDING, BUT NOT
% LIMITED TO, THE IMPLIED WARRANTIES OF MERCHANTABILITY AND FITNESS
% FOR A PARTICULAR PURPOSE ARE DISCLAIMED. IN NO EVENT SHALL THE
% COPYRIGHT OWNER OR CONTRIBUTORS BE LIABLE FOR ANY DIRECT, INDIRECT,
% INCIDENTAL, SPECIAL, EXEMPLARY, OR CONSEQUENTIAL DAMAGES (INCLUDING,
% BUT NOT LIMITED TO, PROCUREMENT OF SUBSTITUTE GOODS OR SERVICES;
% LOSS OF USE, DATA, OR PROFITS; OR BUSINESS INTERRUPTION) HOWEVER
% CAUSED AND ON ANY THEORY OF LIABILITY, WHETHER IN CONTRACT, STRICT
% LIABILITY, OR TORT (INCLUDING NEGLIGENCE OR OTHERWISE) ARISING IN
% ANY WAY OUT OF THE USE OF THIS SOFTWARE, EVEN IF ADVISED OF THE
% POSSIBILITY OF SUCH DAMAGE.
%%

%%
% This is the Eisvogel pandoc LaTeX template.
%
% For usage information and examples visit the official GitHub page:
% https://github.com/Wandmalfarbe/pandoc-latex-template
%%

% Options for packages loaded elsewhere
\PassOptionsToPackage{unicode}{hyperref}
\PassOptionsToPackage{hyphens}{url}
\PassOptionsToPackage{dvipsnames,svgnames,x11names,table}{xcolor}
%
\documentclass[
  paper=a4,
  ,captions=tableheading
]{scrartcl}
\usepackage{amsmath,amssymb}
% Use setspace anyway because we change the default line spacing.
% The spacing is changed early to affect the titlepage and the TOC.
\usepackage{setspace}
\setstretch{1.2}
\usepackage{iftex}
\ifPDFTeX
  \usepackage[T1]{fontenc}
  \usepackage[utf8]{inputenc}
  \usepackage{textcomp} % provide euro and other symbols
\else % if luatex or xetex
  \usepackage{unicode-math} % this also loads fontspec
  \defaultfontfeatures{Scale=MatchLowercase}
  \defaultfontfeatures[\rmfamily]{Ligatures=TeX,Scale=1}
\fi
\usepackage{lmodern}
\ifPDFTeX\else
  % xetex/luatex font selection
\fi
% Use upquote if available, for straight quotes in verbatim environments
\IfFileExists{upquote.sty}{\usepackage{upquote}}{}
\IfFileExists{microtype.sty}{% use microtype if available
  \usepackage[]{microtype}
  \UseMicrotypeSet[protrusion]{basicmath} % disable protrusion for tt fonts
}{}
\makeatletter
\@ifundefined{KOMAClassName}{% if non-KOMA class
  \IfFileExists{parskip.sty}{%
    \usepackage{parskip}
  }{% else
    \setlength{\parindent}{0pt}
    \setlength{\parskip}{6pt plus 2pt minus 1pt}}
}{% if KOMA class
  \KOMAoptions{parskip=half}}
\makeatother
\usepackage{xcolor}
\definecolor{default-linkcolor}{HTML}{A50000}
\definecolor{default-filecolor}{HTML}{A50000}
\definecolor{default-citecolor}{HTML}{4077C0}
\definecolor{default-urlcolor}{HTML}{4077C0}
\usepackage[margin=2.5cm,includehead=true,includefoot=true,centering,]{geometry}
\usepackage{listings}
\newcommand{\passthrough}[1]{#1}
\lstset{defaultdialect=[5.3]Lua}
\lstset{defaultdialect=[x86masm]Assembler}
% add backlinks to footnote references, cf. https://tex.stackexchange.com/questions/302266/make-footnote-clickable-both-ways
\usepackage{footnotebackref}
\setlength{\emergencystretch}{3em} % prevent overfull lines
\providecommand{\tightlist}{%
  \setlength{\itemsep}{0pt}\setlength{\parskip}{0pt}}
\setcounter{secnumdepth}{-\maxdimen} % remove section numbering
\ifLuaTeX
  \usepackage{selnolig}  % disable illegal ligatures
\fi
\IfFileExists{bookmark.sty}{\usepackage{bookmark}}{\usepackage{hyperref}}
\IfFileExists{xurl.sty}{\usepackage{xurl}}{} % add URL line breaks if available
\urlstyle{same}
\hypersetup{
  pdftitle={Handin 2 - Disys},
  pdfauthor={Timmi Andersen 202105859, Thea Hende 202105228, Helena Cooper 201906086},
  hidelinks,
  breaklinks=true,
  pdfcreator={LaTeX via pandoc with the Eisvogel template}}
\title{Handin 2 - Disys}
\author{Timmi Andersen 202105859, Thea Hende 202105228, Helena Cooper
201906086}
\date{21/09/2023}



%%
%% added
%%

%
% for the background color of the title page
%
\usepackage{pagecolor}
\usepackage{afterpage}
\usepackage{tikz}
\usepackage[margin=2.5cm,includehead=true,includefoot=true,centering]{geometry}

%
% break urls
%
\PassOptionsToPackage{hyphens}{url}

%
% When using babel or polyglossia with biblatex, loading csquotes is recommended
% to ensure that quoted texts are typeset according to the rules of your main language.
%
\usepackage{csquotes}

%
% captions
%
\definecolor{caption-color}{HTML}{777777}
\usepackage[font={stretch=1.2}, textfont={color=caption-color}, position=top, skip=4mm, labelfont=bf, singlelinecheck=false, justification=raggedright]{caption}
\setcapindent{0em}

%
% blockquote
%
\definecolor{blockquote-border}{RGB}{221,221,221}
\definecolor{blockquote-text}{RGB}{119,119,119}
\usepackage{mdframed}
\newmdenv[rightline=false,bottomline=false,topline=false,linewidth=3pt,linecolor=blockquote-border,skipabove=\parskip]{customblockquote}
\renewenvironment{quote}{\begin{customblockquote}\list{}{\rightmargin=0em\leftmargin=0em}%
\item\relax\color{blockquote-text}\ignorespaces}{\unskip\unskip\endlist\end{customblockquote}}

%
% Source Sans Pro as the default font family
% Source Code Pro for monospace text
%
% 'default' option sets the default
% font family to Source Sans Pro, not \sfdefault.
%
\ifnum 0\ifxetex 1\fi\ifluatex 1\fi=0 % if pdftex
    \usepackage[default]{sourcesanspro}
  \usepackage{sourcecodepro}
  \else % if not pdftex
    \usepackage[default]{sourcesanspro}
  \usepackage{sourcecodepro}

  % XeLaTeX specific adjustments for straight quotes: https://tex.stackexchange.com/a/354887
  % This issue is already fixed (see https://github.com/silkeh/latex-sourcecodepro/pull/5) but the
  % fix is still unreleased.
  % TODO: Remove this workaround when the new version of sourcecodepro is released on CTAN.
  \ifxetex
    \makeatletter
    \defaultfontfeatures[\ttfamily]
      { Numbers   = \sourcecodepro@figurestyle,
        Scale     = \SourceCodePro@scale,
        Extension = .otf }
    \setmonofont
      [ UprightFont    = *-\sourcecodepro@regstyle,
        ItalicFont     = *-\sourcecodepro@regstyle It,
        BoldFont       = *-\sourcecodepro@boldstyle,
        BoldItalicFont = *-\sourcecodepro@boldstyle It ]
      {SourceCodePro}
    \makeatother
  \fi
  \fi

%
% heading color
%
\definecolor{heading-color}{RGB}{40,40,40}
\addtokomafont{section}{\color{heading-color}}
% When using the classes report, scrreprt, book,
% scrbook or memoir, uncomment the following line.
%\addtokomafont{chapter}{\color{heading-color}}

%
% variables for title, author and date
%
\usepackage{titling}
\title{Handin 2 - Disys}
\author{Timmi Andersen 202105859, Thea Hende 202105228, Helena Cooper
201906086}
\date{21/09/2023}

%
% tables
%

%
% remove paragraph indention
%
\setlength{\parindent}{0pt}
\setlength{\parskip}{6pt plus 2pt minus 1pt}
\setlength{\emergencystretch}{3em}  % prevent overfull lines

%
%
% Listings
%
%


%
% general listing colors
%
\definecolor{listing-background}{HTML}{F7F7F7}
\definecolor{listing-rule}{HTML}{B3B2B3}
\definecolor{listing-numbers}{HTML}{B3B2B3}
\definecolor{listing-text-color}{HTML}{000000}
\definecolor{listing-keyword}{HTML}{435489}
\definecolor{listing-keyword-2}{HTML}{1284CA} % additional keywords
\definecolor{listing-keyword-3}{HTML}{9137CB} % additional keywords
\definecolor{listing-identifier}{HTML}{435489}
\definecolor{listing-string}{HTML}{00999A}
\definecolor{listing-comment}{HTML}{8E8E8E}

\lstdefinestyle{eisvogel_listing_style}{
  language         = java,
  numbers          = left,
  xleftmargin      = 2.7em,
  framexleftmargin = 2.5em,
  backgroundcolor  = \color{listing-background},
  basicstyle       = \color{listing-text-color}\linespread{1.0}%
                      \lst@ifdisplaystyle%
                      \small%
                      \fi\ttfamily{},
  breaklines       = true,
  frame            = single,
  framesep         = 0.19em,
  rulecolor        = \color{listing-rule},
  frameround       = ffff,
  tabsize          = 4,
  numberstyle      = \color{listing-numbers},
  aboveskip        = 1.0em,
  belowskip        = 0.1em,
  abovecaptionskip = 0em,
  belowcaptionskip = 1.0em,
  keywordstyle     = {\color{listing-keyword}\bfseries},
  keywordstyle     = {[2]\color{listing-keyword-2}\bfseries},
  keywordstyle     = {[3]\color{listing-keyword-3}\bfseries\itshape},
  sensitive        = true,
  identifierstyle  = \color{listing-identifier},
  commentstyle     = \color{listing-comment},
  stringstyle      = \color{listing-string},
  showstringspaces = false,
  escapeinside     = {/*@}{@*/}, % Allow LaTeX inside these special comments
  literate         =
  {á}{{\'a}}1 {é}{{\'e}}1 {í}{{\'i}}1 {ó}{{\'o}}1 {ú}{{\'u}}1
  {Á}{{\'A}}1 {É}{{\'E}}1 {Í}{{\'I}}1 {Ó}{{\'O}}1 {Ú}{{\'U}}1
  {à}{{\`a}}1 {è}{{\`e}}1 {ì}{{\`i}}1 {ò}{{\`o}}1 {ù}{{\`u}}1
  {À}{{\`A}}1 {È}{{\`E}}1 {Ì}{{\`I}}1 {Ò}{{\`O}}1 {Ù}{{\`U}}1
  {ä}{{\"a}}1 {ë}{{\"e}}1 {ï}{{\"i}}1 {ö}{{\"o}}1 {ü}{{\"u}}1
  {Ä}{{\"A}}1 {Ë}{{\"E}}1 {Ï}{{\"I}}1 {Ö}{{\"O}}1 {Ü}{{\"U}}1
  {â}{{\^a}}1 {ê}{{\^e}}1 {î}{{\^i}}1 {ô}{{\^o}}1 {û}{{\^u}}1
  {Â}{{\^A}}1 {Ê}{{\^E}}1 {Î}{{\^I}}1 {Ô}{{\^O}}1 {Û}{{\^U}}1
  {œ}{{\oe}}1 {Œ}{{\OE}}1 {æ}{{\ae}}1 {Æ}{{\AE}}1 {ß}{{\ss}}1
  {ç}{{\c c}}1 {Ç}{{\c C}}1 {ø}{{\o}}1 {å}{{\r a}}1 {Å}{{\r A}}1
  {€}{{\EUR}}1 {£}{{\pounds}}1 {«}{{\guillemotleft}}1
  {»}{{\guillemotright}}1 {ñ}{{\~n}}1 {Ñ}{{\~N}}1 {¿}{{?`}}1
  {…}{{\ldots}}1 {≥}{{>=}}1 {≤}{{<=}}1 {„}{{\glqq}}1 {“}{{\grqq}}1
  {”}{{''}}1
}
\lstset{style=eisvogel_listing_style}

%
% Java (Java SE 12, 2019-06-22)
%
\lstdefinelanguage{Java}{
  morekeywords={
    % normal keywords (without data types)
    abstract,assert,break,case,catch,class,continue,default,
    do,else,enum,exports,extends,final,finally,for,if,implements,
    import,instanceof,interface,module,native,new,package,private,
    protected,public,requires,return,static,strictfp,super,switch,
    synchronized,this,throw,throws,transient,try,volatile,while,
    % var is an identifier
    var
  },
  morekeywords={[2] % data types
    % primitive data types
    boolean,byte,char,double,float,int,long,short,
    % String
    String,
    % primitive wrapper types
    Boolean,Byte,Character,Double,Float,Integer,Long,Short
    % number types
    Number,AtomicInteger,AtomicLong,BigDecimal,BigInteger,DoubleAccumulator,DoubleAdder,LongAccumulator,LongAdder,Short,
    % other
    Object,Void,void
  },
  morekeywords={[3] % literals
    % reserved words for literal values
    null,true,false,
  },
  sensitive,
  morecomment  = [l]//,
  morecomment  = [s]{/*}{*/},
  morecomment  = [s]{/**}{*/},
  morestring   = [b]",
  morestring   = [b]',
}

\lstdefinelanguage{XML}{
  morestring      = [b]",
  moredelim       = [s][\bfseries\color{listing-keyword}]{<}{\ },
  moredelim       = [s][\bfseries\color{listing-keyword}]{</}{>},
  moredelim       = [l][\bfseries\color{listing-keyword}]{/>},
  moredelim       = [l][\bfseries\color{listing-keyword}]{>},
  morecomment     = [s]{<?}{?>},
  morecomment     = [s]{<!--}{-->},
  commentstyle    = \color{listing-comment},
  stringstyle     = \color{listing-string},
  identifierstyle = \color{listing-identifier}
}

%
% header and footer
%
\usepackage[headsepline,footsepline]{scrlayer-scrpage}

\newpairofpagestyles{eisvogel-header-footer}{
  \clearpairofpagestyles
  \ihead*{Handin 2 - Disys}
  \chead*{}
  \ohead*{21/09/2023}
  \ifoot*{Timmi Andersen 202105859, Thea Hende 202105228, Helena Cooper
201906086}
  \cfoot*{}
  \ofoot*{\thepage}
  \addtokomafont{pageheadfoot}{\upshape}
}
\pagestyle{eisvogel-header-footer}



%%
%% end added
%%

\begin{document}

%%
%% begin titlepage
%%
\begin{titlepage}
\newgeometry{top=2cm, right=4cm, bottom=3cm, left=4cm}
\tikz[remember picture,overlay] \node[inner sep=0pt] at (current page.center){\includegraphics[width=\paperwidth,height=\paperheight]{images/background9.pdf}};
\newcommand{\colorRule}[3][black]{\textcolor[HTML]{#1}{\rule{#2}{#3}}}
\begin{flushleft}
\noindent
\\[-1em]
\color[HTML]{5F5F5F}
\makebox[0pt][l]{\colorRule[435488]{1.3\textwidth}{4pt}}
\par
\noindent

% The titlepage with a background image has other text spacing and text size
{
  \setstretch{2}
  \vfill
  \vskip -8em
  \noindent {\huge \textbf{\textsf{Handin 2 - Disys}}}
    \vskip 2em
  \noindent {\Large \textsf{Timmi Andersen 202105859, Thea Hende
202105228, Helena Cooper 201906086} \vskip 0.6em \textsf{21/09/2023}}
  \vfill
}


\end{flushleft}
\end{titlepage}
\restoregeometry
\pagenumbering{arabic} 

%%
%% end titlepage
%%

% \maketitle


\hypertarget{handin-2---disys}{%
\section{Handin 2 - Disys}\label{handin-2---disys}}

By Helena Cooper, Timmi Andersen and Thea Hende (202105228).

\hypertarget{exercise-4.6---implementing-a-peer-to-peer-ledger}{%
\subsection{Exercise 4.6 - Implementing a Peer-to-Peer
ledger}\label{exercise-4.6---implementing-a-peer-to-peer-ledger}}

\hypertarget{description-of-how-we-implemented-the-system}{%
\subsubsection{\texorpdfstring{\textbf{Description of how we implemented
the
system}}{Description of how we implemented the system}}\label{description-of-how-we-implemented-the-system}}

In our system we have defined three different kinds of structs, the
\passthrough{\lstinline!Peer!}, \passthrough{\lstinline!ConnectedPeer!}
and \passthrough{\lstinline!MessageStruct!} each serving different
purposes.

First the \passthrough{\lstinline!Peer!} struct is our way of
representing a Peer in a Peer-to-Peer network, and it contains all
relevant information. First off this is a
\passthrough{\lstinline!Name!}, an \passthrough{\lstinline!Ip!} and a
\passthrough{\lstinline!Port!} used to identify the peer. Furthermore it
is a \passthrough{\lstinline!Ledger!}, and a set containing all Peers
the respective Peer is connected to, \passthrough{\lstinline!Peers!}.
Therefore if we were to draw the network, each node would be represented
by an instance of the \passthrough{\lstinline!Peer!} struct. Next up the
\passthrough{\lstinline!ConnectedPeer!} struct represents a connection
made to a peer. This struct is therefore the type that is held in an
array in each Peer's \passthrough{\lstinline!Peers!} field. Note that we
could have chosen not to have this struct since it is very similar to
the \passthrough{\lstinline!Peer!} struct. However we have chosen to
include this to distinguish between \passthrough{\lstinline!Peers!} in
the network and the peers a peer has made a connection to. In this way
we also do not have to worry about the set of connected peers in the
\passthrough{\lstinline!ConnectedPeer!} struct. Finally we have the
\passthrough{\lstinline!MessageStruct!} struct which represents a
message sent between peers over the network. Therefore this contains a
\passthrough{\lstinline!Message!} describing either a request or
response from a peer as a \passthrough{\lstinline!string!}, and some
\passthrough{\lstinline!Data!} that the peer may use depending on the
message.

Next we can take a look on the \passthrough{\lstinline!Connect!},
\passthrough{\lstinline!serverHandling!} and
\passthrough{\lstinline!clientHandling!} methods. As a peer has the
functionality of both a client and a server, we have separated this
functionality into two methods, each handling one of the cases. When a
peer starts up, in our case it is called
\passthrough{\lstinline!Connect!}, we should therefore both do client
and server stuff. Here we spawn one thread for server functionality and
one for client functionality such that they can run concurrently. In
\passthrough{\lstinline!clientHandling!} we do client stuff, which is
first to attempt to make a connection to a given
\passthrough{\lstinline!port!}. Depending on whether or not a connection
was made succesfully we print a message of what happened. If we have a
connection to another peer we request it to send us its set of connected
peers using a \passthrough{\lstinline!MessageStruct!} containing the
message \passthrough{\lstinline!"sendPeersRequest"!}. The sending of
requests over the network is handle by the
\passthrough{\lstinline!sendRequest!} method. At last we can receive
responses on our connection using the
\passthrough{\lstinline!handleResponse!} function.

In \passthrough{\lstinline!serverHandling!} we do server stuff, which
first is to start listening on a random port. Next we make sure that the
peer we are working with is in its own set of
\passthrough{\lstinline!Peers!}. This is to make sure that whenever we
are asked to pass our set of peers to another peer that peer can add us
to its own set. Then we run an infinite loop, where we can accept
connections, and if we get connections we can handle requests coming
from that connection through the \passthrough{\lstinline!handleRequest!}
helper function.

To handle requests and responses we have implemented a total of four
helper functions, namely \passthrough{\lstinline!sendRequest!},
\passthrough{\lstinline!sendResponse!},
\passthrough{\lstinline!handleRequest!} and
\passthrough{\lstinline!handleResponse!}. In
\passthrough{\lstinline!sendRequest!} and
\passthrough{\lstinline!sendResponse!} we send either a request or
response over the network by using \passthrough{\lstinline!gob!} and an
\passthrough{\lstinline!encoder!}. Furthermore
\passthrough{\lstinline!handleRequest!} and
\passthrough{\lstinline!handleResponse!} decodes a message sent with
\passthrough{\lstinline!gob!} and reacts according to the
\passthrough{\lstinline!Message!} string held in the
\passthrough{\lstinline!MessageStruct!}. In case of
\passthrough{\lstinline!handleRequest!} we have three different possible
requests. First \passthrough{\lstinline!sendPeersRequest!} handles the
case where a newly connected peer asks for the set of peers from the
peer already in the network. Here we create a new
\passthrough{\lstinline!MessageStruct!}, convert all
\passthrough{\lstinline!ConnectedPeers!} held in the current peer to an
array of strings (using the
\passthrough{\lstinline!prepareConnectedPeerToSend!} helper function),
and sends this as a reponse. Next ``joinRequest'' handles the case where
a newly connected peer tells another peer that it has connected to the
network. Here we create a \passthrough{\lstinline!ConnectedPeer!} from
the \passthrough{\lstinline!Data!} array using the
\passthrough{\lstinline!makingConnectedPeer!} helper function, and then
adds this to peer's map of peers, \passthrough{\lstinline!Peers!}. At
last \passthrough{\lstinline!transactionRequest!} handles the case where
a \passthrough{\lstinline!Transaction!} was sent to a peer. Here we
first create a \passthrough{\lstinline!Transaction!} from the
information held in \passthrough{\lstinline!Data!} and calls the
\passthrough{\lstinline!Transact!} method on the peer´s ledger. In case
of \passthrough{\lstinline!handleResponse!} we currently have one type
of response we can handle, namely
\passthrough{\lstinline!sendPeersResponse!}. This handles the case,
where a peer sends it set of peers to a newly connected peer in the
network. Here we run through the \passthrough{\lstinline!Data!} array
and unfold each set of three strings in the array to one
\passthrough{\lstinline!ConnectedPeer!}, and we add this to newly
connected peer´s set of peers. Because we assume the peer we connected
to has a connection to all other peers, then our current peer should now
be connected to all other peers as well. At last we prepare a message to
tell all other peers that we joined the network, through the
\passthrough{\lstinline!"joinRequest"!}. This message is then flooded
across the network.

At last the \passthrough{\lstinline!Peer!} can access two other
functions, namely \passthrough{\lstinline!FloodMessage!} and
\passthrough{\lstinline!FloodTransaction!}. In
\passthrough{\lstinline!FloodMessage!} a peer can flood a message to all
other peers in the network, which is done by iterating through the set
of connected peers and making a temporary connection in where we send
the message. Furthermore in \passthrough{\lstinline!FloodTransaction!}
we can flood a transaction to all peers in the network by making a
\passthrough{\lstinline!MessageStruct!} where we input the information
from the \passthrough{\lstinline!Transaction!} into the
\passthrough{\lstinline!Data!} array.

\hypertarget{description-of-how-we-tested-the-system}{%
\subsubsection{\texorpdfstring{\textbf{Description of how we tested the
system}}{Description of how we tested the system}}\label{description-of-how-we-tested-the-system}}

To test the system we have made the \passthrough{\lstinline!handin.go!}
file as asked, where we create NUMBER peers,
\passthrough{\lstinline!p1!} through \passthrough{\lstinline!pn!} and
connect all these to each other by calling
\passthrough{\lstinline!pi.Connect!} with the
\passthrough{\lstinline!addr!} and \passthrough{\lstinline!port!} of the
previous peer, \passthrough{\lstinline!p(i-1)!}. THen\ldots{}

\hypertarget{eventual-consistency}{%
\subsubsection{\texorpdfstring{\textbf{Eventual
consistency}}{Eventual consistency}}\label{eventual-consistency}}

Right now our network has eventual consistency because we allow all
transaction to go through. This means that all peers will eventually
hear all messages resulting in every peers ledger being identical. If a
transaction were to be rejected due to a account going a

\hypertarget{exercise-5.1---one-time-pad-theory}{%
\subsection{Exercise 5.1 - One-time pad
theory}\label{exercise-5.1---one-time-pad-theory}}

\hypertarget{question-1}{%
\subsubsection{Question 1}\label{question-1}}

In this case the employees salary will be represented with maximum 20
bits assuming he earns less than 1 million kroner. He can at most earn
\(999.999\) kr. which is represented as the following in binary:

\[
011110100001000111111
\]

The most significant bit (bit 20 furthest left) will be a \(0\) in the
payment order. If he wanted to receive an extra million in is next
salary he should flip the most significant bit to:

\[
111110100001000111111
\]

The only issue now is that the employee dosn't see the real payment
order but the encrypted payment order. The ciphertext that he sees is
generated by a random secret key so he cant actually be sure that the
most significant bit stays a one. If the secret key also has a \(1\) as
the most significant bit he will receive the same salary as normal and
if it's a \(0\) he will be a millionaire.

\hypertarget{question-2}{%
\subsubsection{Question 2}\label{question-2}}

The issue is not a confidentiality problem because the employee actually
cannot read the message only the encryption. The real issue is
authenticity because the information has been modified by unautherised
people.

\hypertarget{question-3}{%
\subsubsection{Question 3}\label{question-3}}

As stated we have an authenticity problem and not a confidentiality
problem. If done correctly one-time pad is considered unbreakable. The
message cannot be deciphered unless you have the secret-key. The issue
here is that the employee is able to tamper with the ciphertext which
may result in a better salary or may result in a worse salary. One-time
pad insures confidentiality and not authenticity.

\hypertarget{question-4}{%
\subsubsection{Question 4}\label{question-4}}

Assumming the adversary knows that the original bit \(m_i=0\) with
probability .

\end{document}
